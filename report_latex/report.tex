\documentclass[12pt]{report}
%\usepackage[a5paper]{geometry}
\usepackage[utf8x]{inputenc}
\usepackage{lingmacros}
\usepackage{tree-dvips}
\usepackage{ragged2e}
\usepackage{multirow}
\usepackage{placeins}
\usepackage{caption}
\usepackage[framemethod=tikz]{mdframed}
\usepackage{graphicx}
\graphicspath{ {./images/} }
\usepackage{listings}
\lstset
{
    language=Matlab,
    numbers=left,
    stepnumber=1,
    showstringspaces=false,
    tabsize=8,
    breaklines=true,
    breakatwhitespace=false,
    xleftmargin=1em,
    %frame=single,
    numbersep=5pt,
}


\newcommand\mylstcaption{}

\surroundwithmdframed[
hidealllines=true,
middleextra={
  \node[anchor=west] at (O|-P)
    {\lstlistingname~\thelstlisting\  (Cont.):~\mylstcaption};},
secondextra={
  \node[anchor=west] at (O|-P)
    {\lstlistingname~\thelstlisting\  (Cont.):~\mylstcaption};},
splittopskip=2\baselineskip
]{lstlisting}


\begin{document}

\title{Comparing Algorithms for Finding the Closest Pair of Points in a 2D Plane}
\author{David Corbin, Mitchell Harvey and Kyu-Hyeon Lee\\COMP 422, Grove City College}
\date{November 10, 2019}
\maketitle

\section*{Introduction}

The problem we are trying to solve is given $n$ points in metric space,
we want to find a pair of points with the smallest Euclidean distance between the 
points (including points at the same position).

\begin{center}
\includegraphics[width=10cm]{diagram4}
\\\textbf{Figure 1.} The solution to finding the two closest points in a 2D plane.
\end{center}
The closest pair problem is an interesting algorithmic problem with many practical applications. 
For one, air traffic control systems use the closest pair of points algorithm and similar algorithms to direct 
planes on the ground and in the air. The problem itself comes from the field of computational 
geometry, the branch of computer science devoted to the study of algorithms that can be stated in terms of geometry.


\subsection*{Finding the Distance Between Two Points}

To find the distance between two points on a 2D plane, we use the distance formula: 

$$d = \sqrt{(x_2 - x_1)^2 + (y_2 - y_1)^2}$$
\\Geometrically, the distance formula is the length of the hypotenuse of a triangle.


\section*{Brute Force Algorithm}

The straightforward solution is to check the distance between every possible pair of points and find the shortest distance. 
This can be done in a brute force way with a nested for loop.
\\
\\
\textbf{Input:} An array of points; each point having an $x$ and $y$ value
\\
\textbf{Output:} The two points with the smallest distance between them

\subsection*{Pseudocode for the Brute Force Algorithm}

\renewcommand\mylstcaption{Pseudocode for the Divide and Conquer Algorithm}
\begin{lstlisting}[language=Python, caption=\mylstcaption, label=lst:c1]
findClosestPoints(points[]):
	minDistance = Infinity
	closestA = null
	closestB = null
	for a in (0..points.len-1):
		for b in (a+1..points.len-1):
			d = dist(points[a], points[b])
			if (d < minDistance):
				minDistance = d
				closestA = points[a]
				closestB = points[b]
	return [closestA, closestB]
\end{lstlisting}



\subsection*{Explanation}

The algorithm starts by defining the minimum distance to infinity (on line 2) so that the first two points compared will become 
the minimum distance. The two closest points, \textit{closestA} and \textit{closestB}, are set to null initially because there are no points checked before the algorithm starts. 
\\\\
The first for loop (on line 5) loops through the array of points starting with the first element in the array. 
The most naive brute force solution would go through all the points in the array and check all points in the 
array starting at the beginning of the array each time. In this case, you would check the distance between 
point A and B and would check the distance between point B and A. Of course, those distances are the same 
and checking both directions does a significant amount of extra work. This would result in exactly $n^2-n$ 
operations or $\theta{(n^2-n)}$ runtime. An optimization would be not checking the distance between two 
points that have already been checked as shown in the pseudocode above. 
\\\\
The inner loop on line 6 loops through all the remaining elements to check the distance between each 
remaining point in the 2D plane. You can see that we are not checking the distance between points that 
have already been checked. This slight change results in half the number of checked elements or $(n^2-n)/2$
 operations although this is still an order of $n^2$ algorithm.
\\\\
On line 7, we check the distance between the points with the distance formula. This is a constant 
time operation. The remaining lines of the inner for loop check if the distance is less than the 
previous minimum distance and update the minimum distance if it is.

\subsection*{Runtime of the Brute Force Algorithm}

This results in a time complexity of $\theta{(n2)}$ . This is because we loop through every point, then inside 
that loop we loop through every other point, and then compare the distance of those two points. The 
brute force algorithm is in place and the memory consumption $\theta{(n)}$.


\section*{Divide and Conquer Solution}

To get a better algorithm, we look to use a divide and conquer approach. We use an algorithm that is recursive, splitting the problem into two equal subproblems.

\subsection*{Summary of the Algorithm}

\begin{enumerate}
 \item Copy the input points into two arrays $A_x$ and $A_y$.
 \item Sort $A_x$ by $x$ coordinates, and $A_y$ by $y$ coordinates.
 \item Divide $A_x$ into two equal subsets, $p_{left}$ and $p_{right}$.
 \item Find some vertical line $mid_x$ so that all the points in $p_{left} <= mid_x$ and all the points in $p_{right} > mid_x$.
 \item Find the minimum distance between the points in each subset, $d_{left}$ and $d_{right}$.
 \item Find the minimum of $d_{left}$ and $d_{right}$ called $d$.
 \item Get the subset of points from $A_y$ within $d$ distance of $mid_x$ called $p_{center}$.
 \item Check the distance of every point in $p_{center}$ to the next seven points based on decreasing $y$ coordinate. Call the minimum distance $d_{center}$.
 \item Return the minimum of $d$ and $d_{center}$.
\end{enumerate}

\subsection*{Explanation}

We first make two arrays containing the same points. We sort the first array by $x$ coordinates, then sort the second array by $y$ coordinates. The reason 
we have both arrays is important later.
\\\\
Note that we can't afford to sort the arrays and subarrays inside the loop and attain our intended time complexity. Thus, we must presort the arrays so that 
we can achieve $O(n$ lg $n)$ time complexity.
\\\\
We then divide the array of points into two equal subarrays and call the vertical line that separates the halves $L$. Then we recursively find the pair of closest 
points in each half, $d_{left}$ and $d_{left}$. 

\begin{center}
\includegraphics[width=9cm]{diagram3}
\\\textbf{Figure 2.} Divide the set of points in the middle and find the closest points in each half.
\end{center}
We then find the minimum distance between the two closest points in each half; $d = min(d_{left}, d_{right})$. However, we’re not done yet because there could be 
points that were split between the two subsets that are closer than the closest points in each subset. Since we know that we already have two points that 
are $d$ units away from each other, we only need to check for points that are within $d$ units away from $L$. We call this subset of points $p_{center}$.

\begin{center}
\includegraphics[width=8cm]{diagram1}
\\\textbf{Figure 3.} We need only check points within $d$ to the left and right of $L$.
\end{center}
We copy from $A_y$ to get the points in $p_{center}$. This way $p_{center}$ is already sorted by $y$ coordinate. Then our subset $p_{center}$ is formed in $\theta{(n)}$ time, which is important 
to achieve $O(n$ lg $n)$ total runtime.
\\\\
Finding the minimum distance by checking all possible distances between points would not let us complete this step in linear time, so we need a better solution. 
It turns out we can actually find this minimum distance in linear time:
\\\\
We already know that the minimum distance we have found so far is $d$, so we only are interested in distances that are shorter than this. Because of this 
qualification, we actually only need to check a maximum of seven distances for each point. We can prove this geometrically as shown in Figure 4. If every 
point in $p_{left}$ and in $p_{right}$ in is exactly $d$ distance apart, then any possible closer points to a point $p$ in $p_{center}$ will be one of the following seven points. 
This is because the seventh point has to be at least $d$ distance away. Using this to our advantage, we only check the next seven points for each point in $p_{center}$. 
This means that this step is completed in $O(n)$ time.
\\\\
After we complete this step we just have to compare $d$ to the smallest distance we found in $p_{center}$ and return the smallest distance.

\begin{center}
\includegraphics[width=10cm]{diagram2}
\\\textbf{Figure 4.} Check a maximum of seven distances for each point.
\end{center}

\subsection*{Runtime of the Divide and Conquer Solution}
We can find the total runtime of the algorithm by first finding the runtime of the recursive steps:

\begin{enumerate}
\item Divide $A_x$ into two equal subsets, $p_{left}$ and $p_{right}$.
\item Find some vertical line $mid_x$ so that all the points in $p_{left} \leq mid_x$ and all the points in $p_{right} > mid_x$.
\item $2T(n/2)$: Recursively find the minimum distance between the points in each subset, $d_{left}$ and $d_{right}$.
\item Find the minimum of $d_{left}$ and $d_{right}$ called $d$.
\item $\theta{(n)}$: Get the subset of points from Ay within d distance of $mid_x$ called $p_{center}$.
\item $O(n)$: Check the distance of every point in $p_{center}$ to the next seven points based on decreasing $y$ coordinate. Call the minimum distance $d_{center}$.
\item Return the minimum of $d$ and $d_{center}$.
\end{enumerate}
We can represent this as a recurrence: $T(n) = 2T(n/2) + O(n)$. Then by using the master method we find that this recursive section has a time complexity of $O(n$ lg $n)$.
\\\\
Then our full algorithm can be broken down into these steps as follows:
\begin{enumerate}
\item Copy the input points into two arrays $A_x$ and $A_y$.
\item $O(n$ lg $n)$: Sort $A_x$ by $x$ coordinates.
\item $O(n$ lg $n)$: Sort $A_y$ by $y$ coordinates.
\item $O(n$ lg $n)$: Recursively find shortest distance of points.
\end{enumerate}
So the total runtime is $O(n$ lg $n)$. We cannot give an accurate big theta bound, since when we check the points in the center, there is no guarantee of how many points we will need to check in $p_{center}$.

\renewcommand\mylstcaption{Pseudocode for the Divide and Conquer Algorithm}
\begin{lstlisting}[language=Python, caption=\mylstcaption, label=lst:c1]
findClosestPoints(points[], ypoints[]):
    // If 3 points or less then just return shortest distance
    if (points.len <= 3):
        minimum = Infinity
        for a in (0..points.len-1):
            for b in (a+1..points.len-1):
                if (a == b): continue
                minimum = min(minimum, dist(a, b))
        return minimum
		
    // Split points into two equal subarrays
    mid = floor((points.len-1) / 2)
    midX = (points[mid].x + points[mid + 1].x) / 2
    leftPoints = points[0..mid]
    rightPoints = points[mid+1..points.len-1]
    // Get points on each side sorted by y coordinate
    leftYPoints, rightYPoints = getYPoints(ypoints, midX)
                
    // Recurse on the subproblems
    leftMin = findClosestPoints(leftPoints, leftYPoints)
    rightMin = findClosestPoints(rightPoints, rightYPoints)

    // Find the min of the left and right minimum distances
    overallMin = min(leftMin, rightMin)

    // get subset of points within overallMin distance 
    // of vertical line at midX (Figure 3)
    centerPoints = getCenterPoints(yPoints, midX, overallMin)
    // Find minimum distance in center points
    centerMin = findClosestCenter(centerPoints)

    // Find minimum overall
    return min(overallMin, centerMin)

    getYPoints(ypoints, midX):
        rightYPoints = []
        leftYPoints = []
        for p in ypoints:
            if p.x <= midX:
                leftYPoints.append(p)
            else:
                rightYPoints.append(p)
	return leftYPoints, rightYPoints


    getCenterPoints(points, centerLine, dist):
        centerPoints = []
        lowerBound = centerLine - dist
        upperBound = centerLine + dist
        for i in (0..points.len):
             p = points[i]
             if p.x >= lowerBound && p.x <= upperBound:
                 centerPoints.append(p)
        return centerPoints

    findClosestCenter(points):
        minimum = Infinity
        for i in (0..points.len):
            // Check the next 7 points if there are >= 7 points
            numCheck = min(7, points.len - i - 1)
            for j in (0..numCheck):
                d = dist(points[i], points[i + j + 1])
                minimum = min(minimum, d)
	return minimum
\end{lstlisting}

\subsection*{Real World Performance}

We implemented the two algorithms in Python and ran them with various input sizes. 
As you can see, there are several orders of magnitude improvement in the runtimes of the clever algorithm. 
It’s clear that even for relatively small in‌‌‌‌‌‌put sizes, the clever algorithm with a runtime of $O(n$ log $n)$
performs significantly better than the $O(n^2)$ brute force solution.


\begin{table}[hbt!]
\centering
\begin{tabular}{lrrrr}
\multicolumn{5}{c}{\textbf{Real World Results}}                                                                                                                                                           \\
\multirow{2}{*}{\textbf{Input Size}} & \multicolumn{2}{c}{\textbf{Brute Force}}                                     & \multicolumn{2}{c}{\textbf{Clever}}                                          \\
                                        & \multicolumn{1}{c}{\textbf{Runtime (s)}} & \multicolumn{1}{c}{\textbf{Compares}} & \multicolumn{1}{c}{\textbf{Runtime (s)}} & \multicolumn{1}{c}{\textbf{Compares}} \\
500                                  & 0.251                                & 124750                                & 0.071                                & 5096                                  \\
1000                                 & 0.753                                & 499500                                & 0.082                                & 11401                                 \\
1500                                 & 1.638                                & 1124250                               & 0.115                                & 18520                                 \\
2000                                 & 2.990                                & 1999000                               & 0.127                                & 26771                                 \\
2500                                 & 4.534                                & 3123750                               & 0.176                                & 34937                                 \\
3000                                 & 6.202                                & 4498500                               & 0.215                                & 45650                                 \\
3500                                 & 8.823                                & 6123250                               & 0.206                                & 52302                                 \\
4000                                 & 11.452                               & 7998000                               & 0.300                                & 59959                                 \\
4500                                 & 16.524                               & 10122750                              & 0.315                                & 71821                                 \\
5000                                 & 20.771                               & 12497500                              & 0.290                                & 81356                                
\end{tabular}
\end{table}


\FloatBarrier
\subsection*{Extensions of the Closest Pair of Points Problem}
While our project only covers finding the closest pair of points over a 2D plane, the algorithm can 
be extended to find the t​​​​​​​​​wo closest points in a $d$-dimensional space. It's trivial to see that the 
naive solution would have a worst case running time of $O(dn^2)$. With the divide and conquer solution, 
the worst case running time is $O(n * (log(n))^{(d-1)})$ for a normal $d$-dimensional space. However, there 
are point sparsity requirements as the number of dimensions increase to avoid ed​ge cases.




\begin{thebibliography}{9}

\bibitem{subsur} 
Subhash Suri.
\textit{Closest Pair Problem}. 
UC Santa Barbara.
\\\texttt{https://sites.cs.ucsb.edu/\textasciitilde suri/cs235/ClosestPair.pdf}
  
\bibitem{kings} 
Carl Kingsford.
\textit{Closest Pair of Points}. 
Carnegie Mellon University.
\\\texttt{https://www.cs.cmu.edu/\textasciitilde ckingsf/bioinfo-lectures/closepoints.pdf}

\bibitem{hadz}
Vassos Hadzilacos.
\textit{Algorithm to find the closest pair of points}. 
University of Toronto.
\\\texttt{http://www.cs.toronto.edu/\textasciitilde vassos/teaching/c73/handouts/}

\bibitem{cormen} 
Thomas H. Cormen, Charles E. Leiserson, Ronald L. Rivest, Clifford Stein.
\textit{Introduction to Algorithms}. 
The MIT Press.

\bibitem{fort} 
Steve Fortune, John Hopcroft.
\textit{ A Note On Rabin’s Nearest-Neighbor Algorithm}. 
Department of Computer Science at Cornell University.

\bibitem{mount} 
David M. Mount.
\textit{Design and Analysis of Computer Algorithms}. 
University of Maryland Department of Computer Science.
\\\texttt{http://www.cs.umd.edu/class/fall2013/cmsc451/Lects/}

\bibitem{purd} 
\textit{Computational Geometry}. 
Purdue University.
\\\texttt{https://www.cs.purdue.edu/homes/ayg/CS251/slides/chap15d.pdf}

\end{thebibliography}

\end{document}
